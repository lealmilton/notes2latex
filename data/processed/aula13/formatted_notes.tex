\documentclass{article}

\usepackage[utf8]{inputenc}
\usepackage{amsmath, amssymb}

\title{Notes on Controllability of Linear Systems}
\author{}
\date{}

\begin{document}

\maketitle

\section*{Controllability}

\subsection*{Definition:}
A system described by
\[
\dot{x} = Ax + Bu
\]
with state variables $x \in \mathbb{R}^n$ and control variables $u \in \mathbb{R}^r$ is said to be \textit{controllable} if, for any initial condition $a \in \mathbb{R}^n$ and any final condition $b \in \mathbb{R}^n$, and for some $T > 0$, it is possible to find a function
\[
u: [0, T] \rightarrow \mathbb{R}^r
\]
such that
\[
x(0) = a \quad \text{and} \quad x(T) = b.
\]

\subsection*{Remarks:}
The quantities $a$ and $b$ are characteristics of the system (the plant).

\subsection*{Examples:}

\textbf{a)} If $B=0$, the system
\[
\dot{x} = Ax
\]
is not controllable. This can be seen by taking
\[
\begin{pmatrix}
a_1 \\
a_2
\end{pmatrix}
= 0
\text{ and }
b \neq 0.
\]

\textbf{b)} If $B=I$ (where $r=n$), the system
\[
\dot{x} = Ax + u
\]
has the following general solution:
% Here the general solution would be written out if it were provided in the content.

\end{document}